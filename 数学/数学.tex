\documentclass[a5paper]{ctexart}
\usepackage{graphicx, url, float}
\usepackage{geometry}
\usepackage{amssymb,amsmath}
\renewcommand{\d}{\mathop{}\!\mathrm{d}}
\newcommand{\e}{\mathrm{e}}
\renewcommand{\i}{\mathrm{i}}
\newcommand{\R}{\mathbb{R}}
\newcommand{\C}{\mathbb{C}}
\newcommand{\N}{\mathbb{N}}
\newcommand{\Z}{\mathbb{Z}}
\newcommand{\arsinh}{\operatorname{arsinh}}
\newcommand{\arcosh}{\operatorname{arcosh}}
\newcommand{\twp} [4] { \\
	\begin{tabular} {*{2}{@{}p{7cm}}}
		A.~#1 & B.~#2
	\end{tabular} \\
	\begin{tabular} {*{2}{@{}p{7cm}}}
		C.~#3 & D.~#4
	\end{tabular}
} 
\geometry{a5paper,left=2cm,right=2cm,top=2.5cm,bottom=2.5cm}

\title{考研数学}
\author{徐白玉}
\date{January 2024}

\begin{document}
	\maketitle
	\thispagestyle{empty}
	\newpage
	\setcounter{page}{1}
	\tableofcontents
	
	\newpage
	\setcounter{page}{1}
	\section{中值定理}
	\subsection*{一、极限的保号性}
	我们从一道题目入手:\\
	\textbf{例:}(《接力题典1800》P14,选择题4)设 $f(x)$ 为连续函数,且 $\lim\limits_{x \to 0} \frac{f(x) - 2}{e^{x^2} - 1} = 2$,则有(\quad)。
	\twp{$x = 0$ 是 $f(x)$ 的极小值点}{$x = 0$ 是 $f(x)$ 的极大值点\\}{$x = 0$ 不是 $f(x)$ 的极值点}{$(0, 2)$ 是 $f(x)$ 的拐点}
	
	\textbf{解析:}由题意,$f(x)$ 在 $x \to 0$ 处的左右导数均存在,则有 $f(0) = 2$。又由洛必达法则得:$f^{'}(0) = 0$,考察 $x \to 0^+$ 和 $x \to 0^-$ 的情况,可知,$x \to 0^+$时,$e^{x^2} - 1 > 0$,故 $f(x) > 2$; $x \to 0^-$时,$e^{x^2} - 1 > 0$,故 $f(x) > 2$。所以 $x = 0$ 为极小值点。
	
	上述问题的最后一步(分析其值为极大值还是极小值)使用了极限的\textbf{保号性}。上述的解题步骤相当于做这种题的模板。类似的还有《接力题典1800》P15的5,6题。
	
	\subsection*{二、如何求渐近线}
	\textbf{水平渐近线}:若$\lim\limits_{x \to \infty} f(x) = A$,则 $y = A$ 为 $f(x)$ 的水平渐近线。
	
	\textbf{铅直渐近线}:若$\lim\limits_{x \to a} f(x) = \infty$ 或 $f(a - 0) = f(a + 0) = \infty$,则 $x = a$ 为 $f(x)$ 的铅直渐近线。
	
	\textbf{斜渐近线}:若$\lim\limits_{x \to \infty} \frac{f(x)}{x} = a(a \ne 0, a \ne \infty)$,$\lim\limits_{x \to \infty} (f(x) - ax) = b$,则$y = ax + b$ 为 $y = f(x)$ 的斜渐近线。
	
	当某个曲线有了水平渐近线或铅直渐近线,则其不可能再有斜渐近线。
	
	\subsection*{三、中值定理的证明问题}
	中值定理的证明题大多数都是构造函数的问题,但是也存在少数的需要泰勒等技巧的隔路题。
	
	\begin{itemize}
		\item 欲证 $f^{(n)}(\xi) = 0$,一般是罗尔定理和拉格朗日定理的综合应用。只需找 $f^{(n - 1)}(a) = f^{(n - 1)}(b)$ 或 $\frac{a + b}{2}$ 即可。
		
		\item 欲证只含有中值 $\xi$ 的结论,一般要构造函数后使用拉格朗日定理。构造的过程中可能涉及添项。一般地,对于跨阶导数出现时,都需要添中间项,如 $f^{''}(x) +f(x) = f^{''}(x) +f^{'}(x) + f(x) - f^{'}(x) $,随后可使用 $e^x$ 与 $f(x)$ 的复合完成还原。
		
		\item 欲证中值 $\xi$ 和端点值同时出现的结论,一般要么是拉格朗日定理,要么是柯西定理。如果能构造出 $\frac{f(b) - f(a)}{b - a}$ 则是前者,构造出 $\frac{f(b) - f(a)}{g(b) - g(a)}$ 则是后者。
		
		\item 欲证含有多个中值 $\xi, \eta, \zeta$ 的结论,一般需要构造函数后使用两次拉格朗日或一次柯西。
		
		\item 如果题目中出现了三个点的同阶函数,如 $f(a), f(b), f(c)$ 或 $f^{'}(a), f^{'}(b), f^{'}(c)$,一般不考虑泰勒定理,只考虑拉格朗日;但若出现三个点的非同阶函数,如$f(a), f^{'}(b), f(c)$,一般就需要考虑泰勒定理,且定理要求在 $x = b$ 处展开。
	\end{itemize}
	
	上述内容的例题可分别参照《接力题典1800》的如下题目:
	\begin{itemize}
		\item P19 解答题1,2,3.
		
		\item P19 解答题6,7,8.
		
		\item P19 解答题9,11,13.
		
		\item P20 解答题14,15,16.
		
		\item P20 解答题18,20,21.
	\end{itemize}
	
	\section{不定积分}
	
	\subsection*{一、常用不定积分结论}
	
	\begin{itemize}
		\item $\int \sin 2x  \d x= \sin^2 x + C$
		
		\item $\int \frac{1}{1 + \cos x} \d x = \tan \frac{x}{2} + C$
		
		\item $\int (\sqrt{x})^{-1} \d x = 2 \sqrt{x} + C$
		
		\item $\int \frac{1}{1 + \sin x} \d x = \int \frac{1}{1 + \cos (x - \frac{\pi}{2})} \d (x - \frac{\pi}{2}) = \tan (\frac{x}{2} - \frac{\pi}{4}) + C$
		
		\item $\int \tan ^2 x \d x = \int \frac{1 - \cos^2 x}{\cos ^2 x} \d x = \tan x - x + C$
	\end{itemize}
	
	\subsection*{二、徐氏口诀}
	
	\begin{center}
		加常切比常$^1$,\\
		减常对上减$^2$。\\
		常减对上加$^3$,\\
		顺序不要变$^4$。
	\end{center}
	
	$1: \int \frac{\d x}{x^2 + a^2} = \frac{1}{a} \arctan \frac{x}{a} + C$
	
	$2: \int \frac{\d x}{x^2 - a^2} = \frac{1}{2a} \ln \frac{x - a}{x + a} + C$
	
	$3: \int \frac{\d x}{a^2 - x^2} = \frac{1}{2a} \ln \frac{a + x}{a - x} + C$
	
	4: 显然2、3中的“ln”函数为分母的平方差形式。因此,a和x的顺序不能变。
	
	
	\begin{center}
		加减常根形式同,\\
		取对出和分母容$^1$$^2$。\\
		常减根是arcsin$^3$,\\
		积分表里一条龙。
	\end{center}

	$1: \int \frac{\d x}{\sqrt{x^2 + a^2}} = \ln (x + \sqrt{x^2 + a^2}) + C$
	
	$2: \int \frac{\d x}{\sqrt{x^2 - a^2}} = \ln (x + \sqrt{x^2 - a^2}) + C$
	
	$3: \int \frac{\d x}{\sqrt{a^2 - x^2}} = \arcsin \frac{x}{a} + C$
	
	\section{定积分}
	
	\subsection*{一、定积分的本质是求极限}
	
	根据定义,定积分的本质是
	\begin{equation*}
		\int_{a}^{b} f(x) \d x = \lim\limits_{\lambda \to 0} \sum_{i = 1}^{n} f(\xi_i) \Delta x_i
	\end{equation*}
	其中 $\lambda = \max(\Delta x_i)$ 。
	
	因此,对于这道题:计算 $\lim\limits_{n \to +\infty} \frac{1^2 +2^2 + \cdots + n^2}{n ^ 3}$,
	
	其做法是:原式 = $\lim\limits_{n \to \infty} \frac{1}{n} \sum \limits_{i = 1}^{n} (\frac{i}{n})^2 = \int_{0}^{1} x^2 \d x = \frac{1}{3}$.
	也就是说,对于这种题,我们要根据以下步骤来做:
	\begin{itemize}
		\item 提一个 $\frac{1}{n}$ 出来。
		\item 化成带 $\sum \limits_{i = 1}^{n}$ 的求和形式。
		\item 令 $x = \frac{i}{n}$,判断积分的上下限。
		\item 列出积分式并计算。
	\end{itemize}
	
	那么现在你已经知道了该如何计算这类题目,我们来做这道题:计算 $\lim\limits_{n \to \infty} (\frac{1}{n + 1} + \frac{1}{n + 2} + \cdots + \frac{1}{n + n})$.\\
	
	\textbf{解:} 原式 = 
	\begin{equation*}
		\lim\limits_{n \to \infty} \frac{1}{n} \sum \limits_{i = 1} ^n(\frac{1}{1 + \frac{i}{n}}) = \int_{0}^{1} \frac{\d x}{1 + x}  = \ln 2.
	\end{equation*}
	
	有时,这种问题还要和夹逼定理联系在一起,例如:求极限
	\begin{equation*}
		\lim\limits_{n \to \infty} \sum_{i = 1}^{n} (\frac{\sin ^2 \frac{i \pi}{n}}{n + \frac{1}{i}}).
	\end{equation*}
   
	对分母使用夹逼定理,有:
	
	\begin{equation*}
		\frac{1}{n + 1} \sum_{i = 1}^{n} \sin ^2 \frac{i \pi}{n} \le \sum_{i = 1}^{n} (\frac{\sin ^2 \frac{i \pi}{n}}{n + \frac{1}{i}}) \le \frac{1}{n} \sum_{i = 1}^{n} \sin ^2 \frac{i \pi}{n}.
	\end{equation*}
	
	又因为当 $n \to \infty$ 时,
	\begin{equation*}
		\lim\limits_{n \to +\infty} \frac{1}{n + 1} \sum_{i = 1}^{n} \sin ^2 \frac{i \pi}{n} =\lim\limits_{n \to +\infty}  \frac{1}{n} \sum_{i = 1}^{n} \sin ^2 \frac{i \pi}{n} = \int_{0}^{1} \sin^2 \pi x \d x = \frac{1}{2}.
	\end{equation*}
	\subsection*{二、定积分的几个重要性质}
	
	\begin{itemize}
		\item $f(x)$ 连续,则 $\int_{-a}^{a} f(x) \d x = \int_{0}^{a} (f(x) + f(-x)) \d x$.
		
		\item $f(x)$ 连续,则 $\int_{0}^{\frac{\pi}{2}} f(\sin x) \d x = \int_{0}^{\frac{\pi}{2}} f(\cos x) \d x$.
		
		\textbf{证明:}令 $x = \frac{\pi}{2} - t$ 即可。
		\item $\int_{0}^{\frac{\pi}{2}} \sin^n x \d x = \int_{0}^{\frac{\pi}{2}} \cos^n x \d x$,
		且有:设 $I_n = \int_{0}^{\frac{\pi}{2}} \sin^n x \d x$,则 $I_n = \frac{n - 1}{n} I_{n - 2}$,$I_0 = \frac{\pi}{2}, I_1 = 1$.\
		
		\textbf{证明:}将一个 $\sin x$ 挪到 $\d$ 的后边,然后分部积分即可。
		
		通过上述公式可得:
		
		\begin{equation*}
			\int_{0}^{\frac{\pi}{2}} \sin^{10} x =  I_{10} = \frac{9}{10}\times \frac{7}{8}\times \frac{5}{6}\times \frac{3}{4}\times \frac{1}{2} I_0  
		\end{equation*}
		
		\begin{equation*}
			\int_{0}^{\frac{\pi}{2}} \sin^{9} x =  I_{9} = \frac{8}{9}\times \frac{6}{7}\times \frac{4}{5}\times \frac{2}{3}\times I_1
		\end{equation*}
		
		\item $f(x)$ 连续,则$\int_{0}^{\pi} f(\sin x) \d x = 2 \int_{0}^{\frac{\pi}{2}} f(\sin x) \d x $.
		
		\textbf{证明:}令 $x = \frac{\pi}{2} - t$ 即可。
		
		\item $f(x)$ 连续,则$\int_{0}^{\pi} x f(\sin x) \d x= \frac{\pi}{2} \int_{0}^{\pi}  f(\sin x) \d x = \pi \int_{0}^{\frac{\pi}{2}}  f(\sin x) \d x$.
		
		\textbf{证明:}令 $x = \pi - t$ 即可。
		
	\end{itemize}
	
	\subsection*{三、广义积分敛散性的判别}
	
	\subsubsection*{1. 基本判别法(定义法)}
	基本法分三步走。即,对于一个反常积分 $\int_{a}^{\infty} f(x) \d x$,
	\begin{itemize}
		\item 求 $f(x)$ 的原函数 $F(x)$;
		\item 计算 $A = \lim\limits_{b \to \infty} F(b) - F(a)$,\textbf{注意},要把该式当作以 $b$ 为参数的函数;
		\item 如果 $A$ 为有限数,则反常积分收敛于 $A$;反之则发散。
	\end{itemize}
	
	\subsubsection*{2. 区域判别法}
	
\end{document}