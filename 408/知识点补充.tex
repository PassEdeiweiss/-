\documentclass[a5paper]{ctexart}
\usepackage{graphicx, url, float}
\usepackage{geometry}
\usepackage{amssymb,amsmath}
\usepackage{ctex}
\usepackage{tikz}
\usetikzlibrary{matrix}
\usetikzlibrary{arrows}
\usepackage{xeCJK}
\usepackage{xcolor}
\usepackage{rotating} % 旋转文本
\setCJKmainfont{思源宋体 CN}
\usetikzlibrary{positioning, shapes.geometric}
\renewcommand{\d}{\mathop{}\!\mathrm{d}}
\newcommand{\e}{\mathrm{e}}
\renewcommand{\i}{\mathrm{i}}
\newcommand{\R}{\mathbb{R}}
\newcommand{\C}{\mathbb{C}}
\newcommand{\N}{\mathbb{N}}
\newcommand{\Z}{\mathbb{Z}}
\newcommand{\arsinh}{\operatorname{arsinh}}
\newcommand{\arcosh}{\operatorname{arcosh}}
\newcommand{\jst}{\sum \limits_{n = 1}^{\infty}}
\newcommand{\mjst}{\sum \limits_{n = 0}^{\infty}}
\newcommand \jx[3]{\lim\limits_{#1 \to #2} #3}
\newcommand \ding[2]{$\mathfrak{Def}$ \textbf{#1}:#2}
\newcommand {\al}{\alpha}
\newcommand {\be}{\beta}
\newcommand {\ga}{\gamma}
\newcommand{\twp} [4] { \\
	\begin{tabular} {*{2}{@{}p{7cm}}}
		A.~#1 & B.~#2
	\end{tabular} \\
	\begin{tabular} {*{2}{@{}p{7cm}}}
		C.~#3 & D.~#4
	\end{tabular}
} 
\geometry{a5paper,left=2cm,right=2cm,top=2.5cm,bottom=2.5cm}

\title{408补充讲义}
\author{Bai\_Yu}
\date{May 2024}

\begin{document}
	
	\maketitle
	\thispagestyle{empty}
	\newpage
	\setcounter{page}{1}
	\tableofcontents
	
	\newpage
	\setcounter{page}{1}
		\section{写在前面}
		
		本补充讲义是针对“王道计算机考研2025”的四本教材:《数据结构》、《计算机组成原理》、《操作系统》、《计算机网络》做的补充讲义,用于补充书上没有提及,但是实际上已经出现在题目中的内容。
		
		对于书上已经出现的内容,如果其对于理解补充内容有重要意义,我会将其补充在书中;如果无意义,我并不会补充。也就是说,本补充讲义所面向的读者是正在使用该教材的人士。在阅读某一块内容之前,比如数据结构的内容,我们希望读者已经通读原教材中“数据结构”的相关知识。
		
		由于本人学识浅薄,所整理的内容可能存在缺漏或错误。欢迎读者批评指正。
		
		
		
		\newpage
		
		\section{数据结构}
		
		\subsection{算法}
		
		\textbf{算法},顾名思义,就是计算的方法。此计算方法不同于计算机中的“计算方法”学科,那个学科探索的是计算机在做计算时怎样提高解精度的问题,而我们常说的“算法”可以看作一种广义的映射。当带着数据信息的数据结构通过算法之后,就会被映射到对应的解。一般地,计算机学科的“算法”有这样的广为接受的定义:
		
		\vspace{12pt}
		
		\ding{2.1.1(算法)}{一个有穷的指令集,当这些指令为解决某一特定任务规定了一个运算序列时,称作一个\textbf{算法}。}

		\vspace{12pt}
				
		由于计算机学科是一个偏实践的学科,其中的定义大都不要求背诵。所以上边这个定义看看就得了。接下来我们看一看算法有什么性质:
		
		\vspace{12pt}
		
		\textbf{2.1.2 算法的性质:}算法具有\textbf{确切性}、\textbf{有穷性}、\textbf{有效性}。
		
		\vspace{12pt}
		
		\textbf{确切性},说的是算法的每一步指令不能有歧义;\textbf{有穷性},说的是算法应在执行有穷步后结束;\textbf{有效性},说的是每一个步骤都可用计算机指令实现。算法有三大性质,正如集合有三大性质(确定性、互异性、无序性)。
		
		有了算法的定义,我们不难联想:当我们要解决某个特定问题时,我们可能会设计出不同而正确的多种算法。这就引出了两个问题:一是如何设计出算法?二是如何评价这些算法的好坏?这两个问题我们在下面两节里重点地聊一聊。
		
		
		\subsubsection{算法的设计}
		
		算法的设计,是计算机专业看家的一门重要学科。但是为了引入数据结构,我在这里还是不得不提一下这方面的内容,希望可以起到一个抛砖引玉的作用。以下的说法假定读者对 C/C++ 程序设计有一定的基础。
		
		\vspace{12pt}
		
		\textbf{2.2.1 暴力算法:}毫不使用技巧,而是直接靠暴力枚举或暴力搜索解决问题的算法。
		
		\vspace{12pt}
		
		这一算法无论是在理解上还是代码书写上都是简单的。我们以排序问题为例:
		
		排序问题:\textbf{有$n$个数,请把这些数按照由小到大的顺序排列。}
		
		这个问题显然可以使用暴力搜索求解,例如,我们可以找到序列中最小的数置于顶端,再找到次小的数置于第二位,以此类推。我们称这样的暴力搜索算法叫\textbf{插入排序},其正确性是显然的。接下来,我们再介绍一种经典的暴力算法:依次对相邻两个元素的值进行两两比较,若发现逆序则交换,使值较大的元素逐渐从前移向后部。你会发现,这样一来,值较小的元素就像气泡浮出水面一样“浮”到了序列的顶端。因此这一算法有一个形象的名字——\textbf{冒泡排序}。其正确性可以使用\textbf{归纳法}证明。读者可以自行证明并完成代码实现。
		
		聪明的你肯定会思考起来,是否存在其他的排序方法呢?答案是肯定的。例如,我们可以将待排序元素分成大小大致相同的两个子集合,分别对两个子集合进行排序,最终将排好序的子集合合并成为所要求的排好序的集合。那么这就涉及到三个过程:分(Divide)、治(Conquer)、合(Combine),这种排序算法叫\textbf{归并排序}。另外,我们把使用了分、治、合思想的算法统称为\textbf{分治算法}。其正确性依然可以使用\textbf{归纳法}证明。
		
		\newpage
		
		\textbf{2.2.2 分治算法:}把规模较大的大问题分为数个规模较小的子问题,然后通过求解子问题一步步求出大问题的解的算法。
		
		容易发现的是,当我们使用分治算法的时候,我们可能需要把 $n\times k$ 规模的问题压缩为 $n$ 个规模为 $k$ 的问题。为了方便理解,我们暂且机械地把求解该问题的算法定义为函数 $f(x)$ ,其中自变量 $x$ 代表问题的规模。那么当我们想要求解 $f(n \times k)$ 的时候,我们就要调用 $n$ 次 $f(k)$。像这种直接或间接地调用自身的算法称为\textbf{递归算法}。如果你是计算机专业的学生,相信你一定写过使用递归求\textbf{整数阶乘}和使用递归求解\textbf{Hanoi 塔问题}的算法。下面是一个使用递归求\textbf{第 $n$ 个 Fibonacci 数}的示例代码:
		
		\begin{verbatim}
			int F(int n)
			{
				if(n == 0)
				return 0;
				else if(n == 1)
				return 1;
				else
				return F(n - 1) + F(n - 2);
			}
		\end{verbatim}
		
		我们再来思考这样的问题:分治算法对于子问题彼此独立的情况具有良好的求解效果。但是当子问题存在重叠部分的时候,相同子问题将被重复计算多次,浪费计算资源。例如上面的递归求解斐波那契数列的函数就存在这样的问题。对于这种问题的解决方案,相信每位读者都有自己的理解。我们在本章的习题中再做讨论。
		
		\newpage
		
		\subsubsection{算法复杂度}
		
		在前面的学习中,我们看到:对于同一个问题,可能有多个算法能够求出正确的解。那么如何来评价这些算法的优劣程度呢?定性的研究显然是不恰当的。我们应当定义一个或多个\textbf{定量}的标准来比较同一问题的不同算法的效率。为此,我们思考以下问题:
		
		对于一个问题,有两个不同的算法A和B。当给定的数据规模为 $n$ 时,算法A需要占用系统内规模为 $\log _2n$ 的存储空间,而算法B需要占用系统内规模为 $n$ 的存储空间。那么显然算法A\textbf{在内存占用上}是优于算法B的。
		
		对于一个问题,有两个不同的算法A和B。当给定的数据规模为 $n$ 时,算法A需要计算 $n$ 次,而算法B需要计算 $n^2$ 次。那么显然算法A\textbf{在计算次数上}是优于算法B的。
		
		为了定量描述上述内容,我们引入算法复杂度的概念。
		
		\vspace{12pt}
		
		\ding{ 2.3.1 (空间复杂度)} {算法中定义的所有量所占的存储空间之和为算法的\textbf{空间复杂度}。}
		
		\ding{ 2.3.2 (时间复杂度)} {算法中的所有语句运行次数之和为算法的\textbf{时间复杂度}。}
		
		\vspace{12pt}
		
		时间复杂度一般为数据规模 $n$ 的函数 $T(n)$。我们知道,在数学中,我们会使用\textbf{渐进表达式} $T(n) = O(f(n))$ 来表示 $T(n)$ 的上界为 $f(n)$, 即:$n\rightarrow \infty, \ \exists k \in \mathbb{N_+},\ s.t. T(n) \le kf(n)$。因此,当我们能够求出数据规模为 $n$ 的算法具有上界 $f(n)$ 时,称\textbf{算法的时间复杂度上界为 $O(f(n))$}。例如,冒泡排序算法的算法复杂度上界为 $O(n^2)$ 。一般地,当我们提到算法复杂度的时候,说的都是时间复杂度。
		
		类似地,我们还可以定义时间复杂度的下界。即:对于函数 $f(n)$ , 若$n\rightarrow \infty, \ \exists k \in \mathbb{N_+},\ s.t. T(n) \ge kf(n)$, 记作$T(n) = \Omega(f(n))$,称\textbf{算法的时间复杂度下界为 $\Omega(f(n))$}。例如,冒泡排序算法的算法复杂度下界为 $\Omega(n^2)$ 。
		
		对于像冒泡排序这样的算法,它具有一个相等的上界和下界,都是 $f(n) = n^2$,这时我们就可以称\textbf{算法的时间复杂度为$\Theta(f(n))$}。即:存在这样的函数 $f(n)$, 使得:
		
		$$
		n\rightarrow \infty, \ \exists k_1, k_2 \in \mathbb{N_+},\ s.t.\ k_1f(n) \le T(n) \le k_2f(n).
		$$
		
		在实际的问题中,我们往往要处理规模很大的数据。因此,考察时间复杂度的下界对于算法优劣程度意义不大。一般地,我们只通过考察复杂度的上界来考察算法的优劣程度。不难理解的是,对于一个算法,其上界函数 $f(n)$ 的无穷大量阶数越高,其算法的劣势越明显。例如, $O(n)$ 的算法就要比 $O(n^2)$ 的算法要好。
		
		而对于空间复杂度的情况,有以下结论:假定现在有一个解决问题 $K$ 的算法,在这一算法中,我们需要建立 $k$ 个规模为 $n$ 的数组。那么该算法的空间复杂度就是 $O(kn)$,当 $k$ 为常数时,我们可以认为其复杂度为 $O(n)$。类似地,如果我们需要建立 $k$ 个规模为 $n^2$ 的二维数组,空间复杂度就是 $O(kn^2)$,当 $k$ 为常数时,我们可以认为其复杂度为 $O(n^2)$。
		
		接下来我们讨论一些算法复杂度的表示问题。
		
		\vspace{12pt}
		
		\textbf{2.3.3 复杂度的性质:}根据数学中上界的定义,算法复杂度具有以下性质:
		
		\textbf{性质1:}$O(kf(n)) = O(f(n)),\ k \in \mathbb{N_+}$,
		
		\textbf{性质2:}$O(f(n))+O(g(n)) = O(f(n) + g(n)) = max\{f(n), g(n)\}$.
		
		性质1是显然成立的。下面证明性质2:设函数 $T(n)$ 的两个上界函数分别为 $f(n)、g(n)$,即:
		\begin{equation*}
			\begin{split}
				n\rightarrow \infty, \ \exists k_1 \in \mathbb{N_+},\ s.t. T(n) \le k_1f(n)\\
				n\rightarrow \infty, \ \exists k_2 \in \mathbb{N_+},\ s.t. T(n) \le k_2g(n)\\
			\end{split}
		\end{equation*}
		
		
		则有:$T(n) = O(f(n) + g(n))$,其中 $k_1, k_2$ 分别为 $k, k'$ 中的最大值。
		
		\vspace{12pt}
		
		\textbf{1.3.4 算法复杂度分析:} 在算法设计中,我们常常需要估计算法的复杂度,以判断是否有更好的改进方法。常见的算法复杂度包括:$O(n!)$,$O(x^n)$(其中$x > 0$),$O(n^\alpha \log n)$(其中$\alpha > 0$),$O(n^\alpha)$(其中$\alpha > 0$),$O(\log n)$,$O(1)$。这里的对数一般以 $2$ 为底。
		
		下面,我们来分析分治算法的复杂度。为了计算分治算法的复杂度,我们需要考虑分治算法进行了多少次运算。
		
		以分治法递归求解 Fibonacci 数列第 $n$ 项的算法为例:我们可以画出递归树,其中树的高度为 $n$。递归的出口(叶子节点)是 $n = 0$ 或 $n = 1$,这些步骤的复杂度是 $O(1)$;对于非叶节点,其值为两个孩子的值加和,复杂度也是 $O(1)$。因此,整个递归算法的复杂度等于整棵递归树中的节点个数,即 $O(2^{n} - 1)$,也就是 $O(2^n)$。
		
		如果你对树、二叉树以及叶子节点的概念不熟悉,可以参考第四章的相关内容。另外,对于分治算法的复杂度计算,我们有以下结论:
		
		假设一个分治算法的每一步能够把一个规模为 $n$ 的大问题分割成 $k$ 个规模为 $\frac{n}{m}$ 的子问题,且分割和合并的总复杂度为 $O(n^i)$,则整体分治算法的复杂度可表示为:
		$$
		T(n) = kT(\frac{n}{m}) + O(n^i) 
		$$
		可以解得:
		$$
		T(n) = 
		\begin{cases}
			\Theta(n^{\log _mk}) &\log_mk > i\\
			\Theta(n^{i}\log n) &\log_mk = i\\
			\Theta(n^{i}) &\log_mk < i\\
		\end{cases}
		$$
		上述结论称为求解分治算法复杂度的\textbf{主方法(Master Method)}。其正确性的证明并不是必须的,但如果你感兴趣,可以自行探索或查找相关材料加以理解。
		
		举例来说
		
\end{document}
